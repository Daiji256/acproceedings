% +++
% latex = "uplatex"
% +++

% pLaTeX の場合(fonts = default, ms, no)
%\documentclass[platex, dvipdfmx, fonts=default]{acproceedings}
% upLaTeX の場合(fonts = default, ms, no)
\documentclass[uplatex, dvipdfmx, fonts=default]{acproceedings}
% luaLaTeX の場合(fonts = default, ms, word, no)
%\documentclass[lualatex, fonts=word]{acproceedings}

% ハイパーリンク用
\usepackage{hyperref}
% pLaTeX,upLaTeX の場合 hyperref のパッチを当てる
\ifdefined\kanjiskip \usepackage{pxjahyper}\fi
% 数式環境用
\usepackage{amsmath}
% 図関係
\usepackage{graphicx, xcolor}
% ダミーテキスト用(予稿作成には不要)
\usepackage{bxjalipsum}
% pLaTeX などのロゴ用(予稿作成には不要)
\usepackage{bxtexlogo}

\titleJP{{\LaTeX}による予稿作成方法}
\titleEN{How to Prepare a Proceedings using {\LaTeX}}
\authorJP{専攻 太郎}
\authorEN{Taro Senkoh}
\major{〇〇工学専攻}
\supervisor{長岡 高志 教授,悠久 清花 准教授}
\abstract{
	\textsf{acproceedings} is the {\LaTeX} document class
	for Thesis Work proceedings.
	It is based on the \textsf{jlreq} class
	and supports {\pLaTeX}, {\upLaTeX} and {\LuaLaTeX}.
	This document is a sample document
	that also explains how to use it.
	Please use it as a template as it is.
	Templates for Microsoft Word are also available.
}
\keywords{Thesis Work, proceedings, {\LaTeX}, \textsf{jlreq}}
\program{XXX-XXX}

\begin{document}
\maketitle

\section{はじめに}

\textsf{acproceedings}は
専攻科特別研究予稿用の{\LaTeX}文書クラスである.
\textsf{jlreq}クラスをベースとしており,
{\pLaTeX},{\upLaTeX},{\LuaLaTeX}をサポートしている.
ここでは,{\LaTeX}による予稿作成方法を説明する.
また,ソースファイルの\texttt{acproceedings-sample.tex}は,
そのまま,テンプレートとして利用できる.
{\LaTeX}ではなくMicrosoft Word用のテンプレートも用意されているため,
自分の好みに合わせて選択するとよい.

\section{動作要件}

\textsf{acproceedings}の動作要件は,
\begin{description}
	\item[{\TeX}フォーマット] {\LaTeXe}
	\item[{\TeX}エンジン]     {\pLaTeX},{\upLaTeX},{\LuaLaTeX}
	\item[依存クラス]         \textsf{jlreq}
	\item[依存パッケージ]     \textsf{expl3},\textsf{xparse},\textsf{l3keys2e}
\end{description}
のように,なっている.
{\pLaTeX},{\upLaTeX}利用時には,
\textsf{plautopatch}パッケージを自動で読み込む.
本文書クラスの推奨環境は
\textbf{\textsf{{\TeX} Live 2018以降}}である.

\section{使い方}

本文書クラスでは,一般的な{\LaTeX}コマンドや環境を利用できる.
レイアウト設定に影響しないコマンド等でなければ,
基本的に自由に利用して構わない.

\subsection{クラス宣言}

以下に示すように,使用エンジンとDVIウェアをオプションで指定し,
\textsf{acproceedings}を宣言すればよい.
また,\texttt{fonts=...}と指定することで,フォントを設定できる.
指定できるフォントを以下に示す.
\begin{description}
	\item[\texttt{default}]
	未指定の場合この設定になる.
	原ノ味フォントとTimes系フォントで構成される.
	\item[\texttt{ms}]
	MSフォントとTimes系フォントで構成される.
	MSフォントとHGフォントが
	インストールされた環境で動作する.
	\item[\texttt{word}]
	{\LuaLaTeX}専用で,
	MSフォントとTimes New Romanで構成され,
	Microsoft Wordと限りなく近い見た目になる.
	MSフォントとHGフォントとTimes New Romanが
	インストールされた環境で動作する.
	\item[\texttt{no}]
	フォントの変更は行われず,
	\textsf{jlreq}のデフォルトフォントとなる.
\end{description}
macOSやLinuxを利用している場合は\texttt{default}を,
Windowsを利用している場合は\texttt{ms}か\texttt{word}を
選択するとよい.

他にも,のどを無くす\texttt{nogutter}や
キャプション等を日本語にする\texttt{japanese}も指定できる.
基本これらは指定しない.

\subsection{タイトルについて}

本文書クラスではタイトル設定用に,
以下のコマンドが容易されている.
このソースファイルを参考に適宜書き換えるとよい.
\begin{description}
	\item[\texttt{\textbackslash{}titleJP}]    日本語タイトル
	\item[\texttt{\textbackslash{}titleEN}]    英語タイトル
	\item[\texttt{\textbackslash{}authorJP}]   日本語著者名
	\item[\texttt{\textbackslash{}authorEN}]   英語著者名
	\item[\texttt{\textbackslash{}major}]      所属専攻
	\item[\texttt{\textbackslash{}supervisor}] 指導教員
	\item[\texttt{\textbackslash{}abstract}]   Abstract
	({\LaTeX}通常の\texttt{abstract}環境は廃止してある)
	\item[\texttt{\textbackslash{}keywords}]   キーワード
	\item[\texttt{\textbackslash{}program}]    プログラム番号(ページ番号用)
\end{description}

\subsection{数式について}

数式を使う場合は\textsf{amsmath}パッケージを読み込むべきである.
文章中に$E=mc^2$のように数式を記述したり,別行立てで,
\begin{align}
	E = mc^2
\end{align}
のように記述できる.
もちろん$\alpha$,$\leq$,$\Re$のように記号等も使える.

\subsection{図について}

図の挿入は\textsf{graphicx}パッケージで行う
(Figure~\ref{fig:sample}).
図のキャプションは図の下に置く.

\begin{figure}[b]
	\centering
	\includegraphics[width=3cm]{example-image-a}
	\caption{This is an example of figure insertion.}
	\label{fig:sample}
\end{figure}

\subsection{表について}

表組みももちろん利用できる(Table~\ref{tbl:sample}).
表のキャプションは表の上に置く.
この表では,表内の文字サイズを\texttt{small}に設定している.

\begin{table}[b!]
	\centering\small
	\caption{This is an example of table insertion.}
	\label{tbl:sample}
	\begin{tabular}{ll}
		\hline\hline
		\multicolumn{1}{c}{Name} &
		\multicolumn{1}{c}{RGB} \\\hline
		Cyan    & $\mathrm{rgb}(0, 255, 255)$ \\
		Magenta & $\mathrm{rgb}(255, 0, 255)$ \\
		Yellow  & $\mathrm{rgb}(255, 255, 0)$ \\\hline
	\end{tabular}
\end{table}

\subsection{参考文献について}

\texttt{thebibliography}環境で参考文献を書く.
デフォルトでは\textbf{参考文献}となる.
他の文字列に置き換えたい場合,
\begin{quote}
	\verb|\renewcommand{\refname}{引用文献}|
\end{quote}
のようにすればよい.

\section{ダミーテキスト}

\subsection{いろは}

\jalipsum[1]{iroha}

\subsection{寿限無}

\jalipsum[1]{jugemu}

\subsection{吾輩は猫である}

\jalipsum[1-4]{wagahai}

\begin{thebibliography}{9}
	\small
	\bibitem{cite:jlreq}
	W3C日本語組版タスクフォース.日本語組版の要件.
	\url{https://www.w3.org/TR/jlreq/},(参照 2021-10-22).
\end{thebibliography}

\end{document}
