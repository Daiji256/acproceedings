% +++
% latex = "uplatex"
% +++

% pLaTeX,upLaTeX の場合(fonts = default, ms, no)
\documentclass[fonts=default, dvipdfmx]{ac-preprint}
% luaLaTeX の場合(fonts = default, ms, word, no)
%\documentclass[fonts=word]{ac-preprint}

\usepackage{hyperref}
\ifdefined\kanjiskip \usepackage{pxjahyper}\fi
\usepackage{amsmath}
\usepackage{siunitx}

\titleJP{専攻科特別研究テーマ名}
\titleEN{English Title of Thesis Work Theme}
\authorJP{専攻 太郎}
\authorEN{Taro Senkoh}
\major{〇〇工学専攻}
\supervisor{長岡 高志 教授,悠久 清花 准教授}
\abstract{
	This template is prepared for your preparation
	of manuscript for Thesis Works Preprint.
	It is available in MS Word and LaTeX.
	It provides instructions: page layout,
	font style, font size and others.
	You may use it to create your own manuscript
	by replacing the relevant text with your own.

	The Abstract should be justified.
	Font should be a $\SI{9}{pt}$ Serif.
	The length should be 200 words or less.
}
\keywords{$\SI{9}{pt}$, Serif}
\program{XXX-XXX}

\begin{document}
\maketitle

\section{はじめに}

専攻科特別研究発表会講演予稿のテンプレートである.
本テンプレートの体裁にしたがって予稿を執筆すること.

ページレイアウトは,
1行の長さを$23$字,1ページを$48$行とする.
また,のどの余白を$\SI{25}{mm}$とし,
2段組(間隔$\SI{7.5}{mm}$)とする.

フォントは通常ウェイト(明朝とSerif),
\textbf{ボールド(太字ゴシック,Bold Serif)}
の2つの組み合わせで構成される.
基本は,通常ウェイトにMS明朝とTimes New Roman,
ボールドにHGゴシックEとTimes New Roman Boldとする.
原ノ味フォントとTimes系フォントで代用しても良い.

タイトル,英文タイトル,著者,専攻(指導教員名),``Keywords:'',
節,小節,小々節は,ボールドとし,それ以外を通常ウェイトとする.

研究テーマは$\SI{16}{pt}$,英文タイトルは$\SI{14}{pt}$,
著者は$\SI{12}{pt}$,専攻(指導教員名)は$\SI{10}{pt}$とする.
節は$\SI{10.5}{pt}$,中央揃えとし,前後に空行をとる.
小節と小々節は$\SI{10}{pt}$,左揃えとし,前に空行をとる.
本文は$\SI{10}{pt}$で記述する.
図表のキャプションは$\SI{9}{pt}$とし,
図表内およびキャプションを英語(Figure,Table)とする.
ページ番号は下部・中央に指定された形式で記載する.

\section{原稿のページ数}

\subsection{専攻科特別研究発表会講演予稿}

1年生は2ページ,2年生は4または6ページとする.
ページの増減は認めない.

\subsection{専攻科特別研究発表会講演予稿の提出}

提出は,PDFファイルとし,
ファイル名を\texttt{AcMe\mbox{}専攻太郎.pdf}のようにする.
\texttt{Me}のところは出身学科(\texttt{Me},\texttt{Ee},
\texttt{Ec},\texttt{Mb},\texttt{Ci})とする.
締め切りを厳守し,各科専攻科委員に提出すること.

\section{原稿の構成}

原稿の基本構成は,
\begin{enumerate}
	\item 諸言,緒論,序論,はじめに
	\item 実験方法,理論など
	\item 結果及び考察など
	\item 結言,結論,おわりに,まとめ,むすび\\
	参考文献,引用文献
\end{enumerate}
とする.節は,指導教員と相談して適宜変更して構わない.
参考文献には節番号を付けない.

\section{まとめ}

原稿は,本書式の指定を守って作成すること.
また,締切を厳守すること.

\begin{thebibliography}{9}
	\small
	\bibitem{cite:1}
	参考文献は,$\SI{9}{pt}$,番号を付けたリストにする.
	\bibitem{cite:2}
	Microsoft. ``Word help \& learning''.
	\url{https://support.microsoft.com/en-us/word/},
	(accessed 2021-10-17).
	\bibitem{cite:3}
	CTAN. ``The Comprehensive {\TeX} Archive Network''.
	\url{https://ctan.org/}, (accessed 2021-10-17).
\end{thebibliography}

\end{document}
