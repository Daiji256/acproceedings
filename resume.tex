% +++
% latex = "lualatex"
% +++

\documentclass[nogutter]{ac-resume}
%\usepackage[ms]{luatexja-preset}
%\usepackage[ms]{pxchfon}

\usepackage{amsmath}
\usepackage{amssymb}
%\usepackage{bm}
\usepackage{siunitx}
\DeclareSIUnit[number-unit-product=]\percent{\char`\%}

\titleJP{専攻科特別研究テーマ名}
\titleEN{English Title of ACEM Research Theme}
\authorJP{専攻 太郎}
\authorEN{Taro Senkoh}
\major{〇〇工学専攻}
\supervisor{長岡 高志 教授,悠久 清花 准教授}
\abstract{
	This template is prepared
	for your preparation of manuscript
	for Thesis Works Papers.
	It provides instructions: page layout, font style,
	size and others.
	You may use it to create your own manuscript
	by replacing the relevant text
	with your own, using ``cut \& paste.''

	The Abstract should be justified.
	Font should be a 10-point Times-New-Roman.
	The length should be 200 words or less.
}
\keywords{Times, Italic, 10pt}
\program{XXX-XXX}

\begin{document}
\maketitle

\section{はじめに}

専攻科特別研究発表会講演予稿の原稿テンプレートです.
本書式に従い作成して下さい.
本文は,上下左右に$\SI{20}{mm}$の余白を設けること.
2段組(間隔$\SI{7.5}{mm}$)とし,
文字数23,行数47とすること.
研究テーマ,氏名,専攻(指導教員名)は,
MSゴシック,ボールド,中央揃えとする.
英文タイトル・英文氏名はTimes New Romanのボールド,
中央揃えとする.

研究テーマは$\SI{16}{pt}$,英文タイトルは$\SI{14}{pt}$,
氏名は和英とも$\SI{12}{pt}$,専攻(指導教員名)は$\SI{10}{pt}$とする.
章タイトルは,MSゴシック,ボールド,$\SI{10.5}{pt}$,
中央揃えとし,前後を1行空ける.
節タイトルはMSゴシック,ボールド,$\SI{10}{pt}$,左揃えとし,
タイトル前を1行空ける.
本文はMS明朝(英数字はTimes New Roman),$\SI{10}{pt}$で記述する.

図表は,図表内およびキャプションを英語とする.
ページ番号はページの下部・中央に付ける
(専攻ごとに形式が異なるので専攻科委員の指示に従うこと).

\section{原稿のページ数}

\subsection{専攻科特別研究発表会講演予稿}

1年生は2ページ,2年生は4または6ページにまとめてください.
ページの増減は認めません.

\subsection{専攻科特別研究発表会講演予稿の提出}

提出(1,2年生対象)は,
PDFファイル(ファイル名は\texttt{xxxxxx.pdf})とし
各科専攻科委員に提出してください.

\section{原稿の構成}

原稿の基本構成は,
\begin{enumerate}
	\item 諸言,緒論,序論,はじめに
	\item 実験方法,理論など
	\item 結果及び考察など
	\item 結言,結論,おわりに,まとめ,むすび\\
	参考文献,引用文献
\end{enumerate}
です.章タイトルは,指導教員と相談して適宜変更して構いません.
引用文献に章番号は付けません.

\section{まとめ}

原稿は,本書式で示したフォントや余白の指定を守って作成すること.

原稿ページ数は学年によって異なるので,それぞれの指示をよく確認しておくこと.

締切を厳守すること.

\begin{thebibliography}{9}
	\small
	\bibitem{cite:1}
	引用文献は,MS明朝(英数字はTimes New Roman),
	9pt,ぶら下げインデント3mm,番号を付してリストにする.
	\bibitem{cite:2}
	Sigurdson, L.~W.,
	``The structure and control of a turbulent reattaching flow'',
	Journal of Fluid Mechanics 298, pp.139-165 (1995).
\end{thebibliography}

\end{document}
